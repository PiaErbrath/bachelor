\chapter{Project Activities}
This chapter contains information about the more the different activities need to be done during the project.

\section{Analysis of Current State \& Requirements}
The current state need to documented and analyzed based on what went wrong and could be improved. Additional the user of the further workflow will be asked about their wishes. Based on that the requirements will be created driven from regulations of \gls{cc}, laws, the improvement and wishes suggestions. Than they will be weighted regarding the factors importance and realization potential.

\section{Research about Tools}
Based on the workflow requirements and additional for signing tools, like costs or amount of accounts, a research about available tools for signing digital documents will be done. Therefore it may be possible to test them. The testing process will be described in the research. After the research an advice will be given which tool for signing should be used. The decision on which tool to used will be done together with the finance leader.

Additional a research may be done about workflow-management system with automation aspects inside. This is done, because of the fact that the worflow should be extensible and maintainable. Also here a decision is to be made for the tool that matches the requirements best. 

\section{Designing of New Workflow}
The new process need to be designed depend on the previous found requirements and regulations from \gls{cc} and the governance. This design should be discussed with persons that should use it later and based on their feedback reworked.

\section{Implementation of New Workflow}
The implementation of the new process should be done regarding the design plan. Additional every new functionality need to be tested regarding a test plan. The process should also be tested by person that may use them later. Based on the test and feedback of non-developer improvements can be done. 