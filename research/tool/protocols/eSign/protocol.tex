\section{Test Protocol eSign live}
\label{sec:esign}
The tool is tested based on the scenario described in the research tools to sign documents electronically. It is done at the 16th April 2018. Furthermore, the \gls{app} for Android is tested.

\subsection{Registration}
To register for  the test version the user need to enter the following data: name, mail, password, company name, company are, phone number and country. In the case a person is already registered he can login with the mail and the password.

\subsection{Uploading Documents}
The systems (\gls{app} and web) allow different document usage:
\begin{itemize}
	\item Web application:
	\begin{enumerate}
		\item Template: Inside the tool the user can create templates for standard documents like Non-Disclosures Agreement. They could be used as much the user want.
		\item Finished document from computer: Therefore, the user clicks on a button and a file chooser pops up, where the user navigates to the storage of the document file and select it.
	\end{enumerate}
	\item \Gls{app}:
	\begin{enumerate}
		\item Choose an existing document file from device: Therefore, the user clicks on a button and a file chooser pops up, where the user navigates to the storage of the document file and select it.
		\item Select from another \gls{app}: The user can select documents file stored in another storage \gls{app}. Inside this the user can navigate as given from the \gls{app}.
		\item Make a image of the document: The \gls{app} offers the option to make a photo from the document need to sign.
		\item Template: Inside the tool the user can create templates for standard documents like Non-Disclosures Agreement. They could be used as much the user want.
	\end{enumerate}
\end{itemize}

\subsection{Sharing of Documents}
To share a document for signing, notification or forwarding just the mail address and the name of the persons should receive the document need to be entered.

\subsection{Select Order of Signing}
In the case there should be an order to sign, this option need to be enabled within the view to add signers and persons that need to be notified. Afterwards, the user can set the order either by adding them based on the order or by drag and drop the different parties.

\subsection{Preparation for Signing}
Before a document could be send to sign, the required fields need to be set via a drag and drop functionality. The fields can be chosen per person who had to fill the field or can be selected in the option of the field. Additionally, each field beside the signature field can be set either as required or as not required.

The following field types are available at eSign Live: signature, initials, date, name, title, company, text field, text area, check box, radio, list and label field.

\subsection{Company Stamp}
The tool does not provide the possibility to add a company stamp directly. It can be done with the attachment of documents, where the description explains that a company stamp is required and set this to required.

\subsection{Signing Documents}
eSign Live offers two options to sign a document:
\begin{itemize}
	\item Click-to-sign: The signer clicks on a button and confirm the agreement to accept this as a signature.
	\item  Capture signature: The signer draw his signature with the mouse or other device in  the area given from the tool to draw. Afterwards, he confirm that he accepts this as signature.
\end{itemize}

\subsubsection{One Person}
In the case the person has an account he will be notified by the tool in the web application or the \gls{app} and with a mail. In the case he goes with the mail, the user clicks on a link and will be forwarded to the document. Otherwise, he can access the document directly by clicking on it. First of all he need to accept the disclosures, this step is not done if the signer is equal to the initiator of the signing process. Next the user clicks on the fields that need to be filled manual, if they are not required he is not indebted to fill them. Afterwards, he generates the signature as described above and finish the signing process for him.

\subsubsection{Multiple Persons}
For each person it is equal to one person. There are two more aspects, that need to mentioned here. The first one is that the signers will be notified to be signed after the previous signer has signed if there exists an order for the signing. The second aspects is that the document can only accessed by one person per time. Thats means if two persons want to sign the document at the same time, it is not possible, because the tool will one of them deny the access until the other has closed the document or finished his signing process.

\subsubsection{Unregistered Person}
In the case the signer is not registered at eSign Live he will only receive a mail with a link. By clicking this link the signer will be redirected to the document and can sign as described for one person.

\subsubsection{Decline Signing}
The declining of the signature is not offered by eSign Live. It could only be done by do not signing the document and in the case there is end data for signing it will be marked as not agreed.

\subsubsection{Features}
eSign Live offers more actions than requested in the test scenario. They will be described below.

\paragraph{In-person signing}

\paragraph{Review before completion} need to be reviewed by the manager before it could be completed

\paragraph{Authentication} mail, sms, Q\&A

\paragraph{Attachments} adding of documents (name and description required so that it is known what is requested)

\paragraph{Disclosures} before real document -> legal

\paragraph{Accept only}

\paragraph{Delegate signing} need to be enabled per user, allows additional security request (mail, name); bt in the top not direct visible; add person in cc

\subsection{Status Report}
- at start side
- report about all in seperate view (sorted by time span, document status)

\subsection{Availability of Document}
- download after signature
- completed with mail
- account webside/ app

\subsection{Personal Impression}