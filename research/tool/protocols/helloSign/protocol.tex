\section{Test Protocol HelloSign} \label{sec:hellosign}
This document is a testing protocol created at the 9th April 2018. The tool HelloSign is tested based on the test scenario described in the research about Tools to Sign Documents Electronically.

\subsection{Registration}
HelloSign offers different plans with several extending functionalities. For the registration of a none free version, the credit card number is requested, due to the fact that the versions runs until it will be terminated. For the testing the free version is used, but this is limited to documents and functionalities. The other test versions would be available for 30th days for free and would be automatically transferred to a paid version after these days.

To get the free version some information are requested:
\begin{itemize}
	\item Name,
	\item Position,
	\item Department,
	\item Company name,
	\item Business area and
	\item A reason why it is used.
\end{itemize}
In the case another version is used, the credit card information need to be given. \newline
The registration in general is simple. First you need to add your mail address, verify it and then select a password.

\subsection{Uploading Documents}
To upload a document several options exist. You can select a file within the document chooser, use a template stored at HelloSign, drag the file in the browser or import the document from a service like Dropbox or Google Drive.

Within the tool the tested document formats \gls{pdf}, Microsoft Word and Open Office Writer was accepted without errors. Moreover, it was possible to selected different formats for one signing process.

\subsection{Sharing of Documents}
To add person, who had to sign the documents is simple. Just add the name of them and insert the mail address in the required fields. At the beginning the process initiator select which parties had to sign. He can choose between those three options: me, me \& others and others. Before the invitation for signing is sent the subject and the body of the mail could be defined.


\subsection{Select Order of Signing}
Also, the selecting of an order for signing is simple. Just activate it with a checkbox and then move the different signing parties in the correct order. The only disadvantage is that multiple parties can not sign at the same order level.

\subsection{Preparation for Signing}
Inside the documents the fields for signing are added manually by drag and drop. The field could be from the following types: signature, initials, textbox, checkbox and sign date. In the case multiple parties need to sign the document, the initiator need to define which field belongs to which party. If the initiator need also to sign the document, he can it do directly after the setting the corresponding fields. 

\subsection{Company Stamp}
This functionality is not testable, because therefore at least the business plan is required, which is not available for testing.

\subsection{Signing Documents}
In the case there exist no saved signature, several options exist to generate one:
\begin{enumerate}
	\item Draw a signature: Within this option the mouse is used to draw the signature.
	\item Type a signature: The user types its name in the field and the tool generates the signature. In the next step the user can select the style of the signature.
	\item Upload an image of a signature: select image, adjust it
	\item Use a smartphone: make picture of signature, mail it to specific mail with certain subject, go ahead as previous step
\end{enumerate}

After adding the signature to the document, the signer need to agree that he accepts the usage of \gls{es}.

\subsubsection{One Person}
The signing of a document is fast. The user receives a mail or is directly by setting up the file requested to sign the document. In the mail a button need to click to open the document. Then the fields need to filled in with the requested data. In the case of a signature either the actions described above need to be performed or if the user signed already the stored signature could be used. It is also similar for the initials.

\subsubsection{Multiple Persons}
Every single person sign the document as described in the part for one person. The only different is, that they receive status information about the actions from the other persons. Moreover, it is not possible to sign against the order, the person, who has to sign later, receives a mail that he has to sign at a certain time and will receive another mail when it is possible.

\subsubsection{Unregistered Person}
The signing process for an unregistered person is also as described in the part above for one person. He receives an invitation to sign the document, enter the data required, select a signature and agree the usage of \gls{es}. Finally, he receives the question if he wants to activate a free account, which could be ignored.

\subsubsection{Decline Signing}
Every party has the option to decline the signing. Therefore, the action need to be selected from the available options of the document. The person is asked to give reason and send the information.

\subsubsection{Features}
In higher versions of HelloSign than the free version some features, like branding or changing of settings are available, which are not tested during the circumstances of using the test versions. 

Moreover, HelloSign provides an \gls{app} for Android and iOS, which is available for free. It provides the functionality to start a signing process. Therefore, the documents can be scanned in as picture or imported from the device the \gls{app} is running on. But it is not possible to sign a document that is assigned to the user of the \gls{app}, if it is not in the initiation process. Also, it is not possible to see that status of document.

\subsection{Status Report}
Inside the web application the user can see the status of a document, if he has an HelloSign account. The status is presented through colors and descriptions. There exists also the functionality to see the status of the document by all parties assigned to the process with a mail report. 

\subsection{Availability of Document}
HelloSign offers for user the option to handle documents. They could be deleted, shared, edited or downloaded from the web application. But all persons assigned to the process receives after completion a mail with the signed documented attached.

\subsection{Personal Impression}
In general need to be said, that HelloSign may be useful, but the need to give payment data to test the tool in a higher version and the not ideal \gls{app} leads to an unsatisfied feeling by using it. The general handling is simple and intuitive, within the functionalities could be tested during the test.