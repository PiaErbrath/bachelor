\section{Test Protocol DocuSign}
Inside this part the testing of the tool DocuSign is documented. The tests are executed based on the definitions of research about Tools to Sign Documents Electronically. It is done at the third and fourth April 2018. 

\subsection{Registration}
The registration for testing the tool is simple. The following data need to be added:
\begin{itemize}
	\item Name,
	\item Email,
	\item Job title,
	\item Company,
	\item Business area,
	\item Amount of employees,
	\item Phone number and
	\item A reason why it is used.
\end{itemize}
Later you need to select a password and confirm it. The testing duration will take 30th days. After that time the signed documents are still available, but a new starting of a signing process is not possible without payment. It also possible to sign free, but that is limited through the number of starting a signing process. 

The normal login is simple, you only need to enter your mail and the selected password. That is the same for the \gls{app}. In the case a person only needs sign a document, no registration is necessary. The person can access the document via a link every time.

\subsection{Uploading Documents}
In the web version a button is available which opens by clicking a document chooser with the directory of the computer the web browser is opened. Through that the navigation is similar to other tools like Microsoft Word. Now the tester can navigate to the test document and select it. Then DocuSign offers the option to select additional files. It is also possible to add a document with the drag-and-drop functionality. Furthermore, a document could be chosen from templates stored at DocuSign. If the \gls{app} is used a photo from the document could be made and this could then be used to sign.

In the case a document is selected and uploaded, several actions could be done with it: Show the document, rename it, compare it with a template, replace or delete the document. DocuSign accepts several document types like \gls{pdf} or Microsoft Words .docx.

\subsection{Sharing of Documents}
The sharing is simple. Only the name and the mail address of the person need to sign is to enter. In the case the person was already requested to sign a document, the contact data is stored by DocuSign in the web version and could be selected with the address book. The \gls{app} uses the contact data from the smartphone.
As a feature different roles could be selected:
\begin{enumerate}
	\item A person which need to sign
	\item A person which sign locally (only in the business version)
	\item A person which receives an information
\end{enumerate}
Moreover, there is the option to add an additional security level. This is done by requesting an access code defined by adding the person to sign. In the case the person now want to sign, he/she need to enter before the code. This code need to be transmitted somehow to this person. Furthermore, it is feasible to remove a person from the list of sharing. 

\subsection{Select Order of Signing}
Within the web application it is possible to select an order for the signing, with the test in the \gls{app} it was not possible. It is simple, either do it with drag-and-drop or enter the number before the contact box. Also, the user can select if a order is necessary or if two or more person need to sign first ante the other persons.

\subsection{Preparation for Signing}
In this action all required fields need to be manual placed in the document. This is quite easy and there are a lot of fields type by default, which can be customized in the business version of the tool. The fields are for example signature, name, mail or date of signing.
Also, it is possible to twist the document if it is necessary. Furthermore, there is the option that on each document file uploaded to sign could be prepared separate. 

\subsection{Company Stamp}
The tool allows the using of stamps. Therefore, this option need to be activated in the settings. When this action is done, a new field type is available by the preparation for the signing action.

In the case the stamp is the first time requested, a window is open with the request to upload a picture of the stamp. This picture can be fit and gets a name. Also there is the option to declare the created stamp as default stamp. 

\subsection{Signing Documents}
Within the step of signing a document several options are possible, each of them is described bellow.

\subsubsection{One Person}
Regarding the fact that not always more than one person need to sign, two options exist. First the person who start the process need to sign or second another person need to sign. Both cases are similar and are basically the same as multiple person would sign the document. The only difference is that int he first option the user will be redirected after the preparation to the signing action. int he other case the user receives an mail. 

\subsubsection{Multiple Persons}
All persons get an invitation via mail to sign the document. If there is a specified order, they receive the mail not until the person(s) need to sign previous have signed it successfully. When all persons have signed the document, all of them will receive a mail with the link to the completed document.

\subsubsection{Unregistered Person}
If the person need to sign the document is not registered at DocuSign, he gets an mail and can access the document via a button or link. Then the person need to agree the usage of an \gls{es} or can select other options like decline or alternatives explained in the features. Next the requested fields need to be filled in. In the case of the signature filed, the person can correct the name and the initials and is allowed to choose between selecting a signature style or drawing the signature with the mouse. Finally, if all required fields are filled in, the signing process could be finished, a windows pop ups with the option to register at DocuSign and a mail is send to all parties when the document is completely signed. 

\subsubsection{Decline Signing}
All persons, who not started the signing process, have the right to decline the signing. Then the complete order of signing is stopped and all parties get a mail with the information about the declining. The person have the option to enter a reason, which is also submitted with the mail.

\subsubsection{Features}
The tool provides several features not requested in the test setup, but are still tested. They will be explained beneath.
\paragraph{Signing in a Local Session}
When the mail is received with the link of the to be signed document and the button is clicked to access it, first an information page about the process is shown. After the clicking the start button the users are leaded through the process. The person need to be signed, accept the using of the \gls{es} and can then fill in the fields. In the case of the signature filed, the person can correct the name and the initials and is allowed to choose between selecting a signature style or drawing the signature with the mouse. Finally, if all required fields are filled in, the signing process could be finished, a mail address is requested to send a copy of complete signed document. 

\paragraph{Print and Sign}
Sometimes a person or company accepts the \gls{es}, but want to sign with the handwritten signature. Therefore, this option exists at DocuSign. The person which had to sign can print the document, sign it with all required fields manual and either upload it again to DocuSign or send with fax back.

The uploading process is simple, just download the document, print and sign it, scan it and finally upload it again. The only disadvantage which is inside this process, is that the person need to sign either should not close the option and do the previous described process directly or need to start the process again. 

In the case the fax option is used the document need to be downloaded. This document is different than the standard document, because it has an additional page containing all information required to fax it back to the tool. 

\paragraph{Assign to Someone Else}
Regarding the fact, that not every time the correct person is assigned to sign a document, DocuSign offers the option to redirect this task, by the person which receive the invitation of signing. This person can refer the document to a different person by adding their name and mail address and gets a copy of the signed document. The other person receives a request to sign the document as described above. Every time the document owner, who started the signing process, is informed about such changes.
It is possible to denied such action by setting a permission at the step where the process is started. 

\paragraph{Reverse Signing Process}
In the case a document is not completely signed, there is the option for the person who started the signing process to stop the process. The document is then voided. All parties involved in the stopped process get an information that the document is voided.

\paragraph{Reports}
A feature of DocuSign is that it creates statistics and reports about the usage of the tool. This means it presents statistics about the amount of documents to be signed or their status. There are more options available, here is only a short overview given.

\subsection{Status Report}
Within DocuSign it is possible to track the status of a document. Several different information exist like at which time some received, looked and signed the document. In this functionality other actions could be triggered like sending the document again, make changes to it or download the current document.

Furthermore, there is the options for those users which have a DocuSign account to see open documents, that are assigned to be signed by them. This view also available in the \gls{app}.

\subsection{Availability of Documents}
After the signing process is completed all persons related to the document receives a notification, that contains a link to the document. Furthermore, the document is stored at DocuSign and can be accessed with the login data of the accounts. Inside the tool they are sorted by different categories e.g. if the person send the document or if the document is completed. The document always can be downloaded or printed in the online document view. This functionality is available with the link send from DocuSign or by accessing from the tool itself.

\subsection{Personal Impression}
The handling of the tool DocuSign is simple. It has a lot of options to sign documents, even in the case one of the involved parties is not allowed to sign electronically, do not want to create an account or pay for it. Also, there are a few nice features available, which at the moment are not requested by \gls{cc}. Moreover, the look and feel is user friendly.