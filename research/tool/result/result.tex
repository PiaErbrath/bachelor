\section{Result}
In this section different tools will be described, their test result presented and the general information of them summarized.

\subsection{Wacom sign pro PDF}
This tool is provided by the company Wacom, which is the customer previously thought for the bachelor project. At the moment there is an \gls{app} available, that signs documents with a \gls{bs} \parencite{wacom2018pdf}. Currently Wacom develops an \gls{sdk} based on the \gls{app}, therefore is no testing possible. The information presented in table \ref{tab:wacom} are from the internet and a product presentation from first March 2018 at the Wacom site in Düsseldorf. 
\begin{table}[h]
	\begin{tabular}{|p{4cm}|p{10cm}|} \hline
		Criterion & Fulfillment \\ \hline
		Usage without license & At the moment not clear. The \gls{app} currently has a free version, but with limited usage.\\ \hline
		Devices independent & The usage of this technology requires hardware at the moment. Therefore, several options exist. Either you have a smart phone with a pen, an iOS phone, a Windows computer with touch recognition or a Windows computer with additional hardware.\\ \hline
		Integration possible & This year a new \gls{sdk} will be published that allows the integration with existing tools. \\ \hline
		Certified & The software uses the \gls{bs} and \gls{aes}. Currently there is no defined status by the government for the \gls{bs}. But in combination with the \gls{aes} it has a legal status. \\ \hline
		Easy usage & Select field(s) to enter data and sign with the pen, press button to agree and then enter mail to sharing. It is not described if a company stamp could be inserted.\\ \hline
		Costs & Unclear, due to the fact that the \gls{sdk} is not published and no pricing is published.\\ \hline
		Accepted documents & \Gls{pdf}\\ \hline
		Amount of accounts & At the moment also unclear due to the fact that the usage requirements are not published. The \gls{app} currently allows more than 5000 in one environment. \\ \hline
	\end{tabular}
	\caption{Summary Wacom sign pro PDF}
	\label{tab:wacom}
\end{table}

\subsection{DocuSign}
This tool is already in usage at \gls{cc}, but not always and with a lot of manual actions and missing things like the company stamp. During the testing a lot of functionalities were figured out. The test protocol is placed in the appendix \ref{sec:docusign}. The table \ref{tab:docusign} below will give a short summary about the test result and other aspects figured out during the research.

\begin{table}[h]
	\begin{tabular}{|p{4cm}|p{10cm}|} \hline
		Criterion & Fulfillment \\ \hline
		Usage without license & Yes, just need to add some information required from the person request the signing like name and mail address for signing. In the case of uploading and managing the documents an account is required.\\ \hline
		Devices independent & Online available, also a \gls{app} is accessible for all \gls{os} of smartphones \parencite{docusign2018mobile}. \\ \hline
		Integration possible & A \gls{sdk} available within the \grqq Business Flex\grqq packet of the provider. Also an integration is possible with Google Drive, but therefore the Google Drive cloud need to be connected to DocuSign \parencite{docusign2018integration,docusign2018formats}. \\ \hline
		Certified & DocuSign satisfies the standards of the \gls{eidas} and more security standards \parencite{docusign2018certificates,docusign2018legal}. \\ \hline
		Easy usage & The fields need to be set manual, it need to be checked if it possible to customize it. Furthermore it is possible to track activities, assign signer, add the company stamp, transfer the signing responsibility, decline the signing and to sign the document manual and scan it then. \\ \hline
		Costs & The provider of DocuSign offers different plans. \Gls{cc} needs at least the \textit{Business Pro} Plan for one user, which costs 480\$ per year, and the \textit{intermediate API} plan with 3420\$ per year. But due to the fact, that there will  e more than one user, the enterprise offerings should be requested. \parencite{docusign2018api,docusign2018user} \\ \hline
		Accepted documents & PDF \& Microsoft Word accepted, Open Office Writer is not accepted. Moreover, the maximum accepted size of a document is 25 \gls{mb}. \parencite{docusign2018formats} \\ \hline
		Amount of accounts & Depending on the license ordered.\\ \hline
	\end{tabular}
	\caption{Summary DocuSign}
	\label{tab:docusign}
\end{table}

\subsection{HelloSign}
The online tool HelloSign is used at the subsidiary company Istana of \gls{cc}. Therefore, it should also taken into account. Not all functionalities could be tested through the fact that for the business version data were requested, the tester was not allowed to give. The result of the test is documented in appendix \ref{sec:hellosign} and a summary of it together with additional information collected from an online research is given in table \ref{tab:hellosign}.
\begin{table}[h]
	\begin{tabular}{|p{4cm}|p{10cm}|} \hline
		Criterion & Fulfillment \\ \hline
		Usage without license & Signing without account possible, but the signed documents could not be viewed online and uploaded if no account exists. It will be send within a mail. Moreover, there is the fact that the free version only allows to send three documents per month. This leads to the result, that a higher version is required \parencite{hellosign2018price}. \\ \hline
		Devices independent & The tool is online available. Furthermore, exists an \gls{app} for Android and iOS, but it has not the same functionalities as the online platform.  \parencite{hellosign2018legal} \\ \hline
		Integration possible & HelloSign has an \gls{api} with different version provided, included there are several functionalities. Moreover, integration exists for Google services like Google Suite and for several other business applications e.g. HubSpot CRM, which used at \gls{cc}. Additionally third party integrations are possible with Zapier, a tool for modeling business processes. \parencite{hellosign2018integration,hellosign2018api} \\ \hline
		Certified & The tool fulfills the requirements of the \gls{eidas} regulations of the \gls{eu} and can be used in the \gls{usa} \parencite{hellosign2018legal} \\ \hline
		Easy usage & It is simple to upload a document and sign it, due to the fact that there is intuitive handling. Not all functionalities like branding and company stamp could be tested, because of the non usable test version of higher plans. The fields required for signing need to be set manual and there is not that much choice of fields. In the case the \gls{app} is used, only the creation of a signing process could be done. \\ \hline
		Costs & For the usage at \gls{cc} two different licenses are required. First the \gls{api} license where at least the version API Gold is need which costs 449\$ per month. The second license is the user license. Therefore the User Business version is required at least with five senders. It costs 40\$ per month. But due to the fact, that more senders are needed the Enterprise version is advisable. Yet the price is not given in the internet\parencite{hellosign2018price,hellosign2018api} .\\ \hline
		Accepted documents & HelloSign accepts several documents format. In the test the formats \gls{pdf}, Microsoft Word and Open Office Writer was accepted. The maximal amount of a pages a document is 500 pages or a file size of 40th \gls{mb} \parencite{hellosign2018documents}. \\ \hline
		Amount of accounts & Inside the business version of HelloSign 5 senders/accounts are involved, for more accounts the provider need to be requested \parencite{hellosign2018price}. \\ \hline
	\end{tabular}
	\caption{Summary HelloSign}
	\label{tab:hellosign}
\end{table}

\subsection{SignDoc}
This tool is provided by Kofax, a partner of \gls{cc}.

\begin{table}[h]
	\begin{tabular}{|p{4cm}|p{10cm}|} \hline
		Criterion & Fulfillment \\ \hline
		Usage without license & \\ \hline
		Devices independent & \\ \hline
		Integration possible & \\ \hline
		Certified & \\ \hline
		Easy usage & \\ \hline
		Costs & \\ \hline
		Accepted documents & \\ \hline
		Amount of accounts & \\ \hline
	\end{tabular}
	\caption{Summary SignDoc}
	\label{tab:signdoc}
\end{table}

\subsection{Adobe Sign}

\begin{table}[h]
	\begin{tabular}{|p{4cm}|p{10cm}|} \hline
		Criterion & Fulfillment \\ \hline
		Usage without license & yes, but for starting license required\\ \hline
		Devices independent & web, app iOS and Android (require certain os version) \\ \hline
		Integration possible & A lot of integration with tools possible, api available \parencite{adobesign2018integration,adobesign2018info} \\ \hline
		Certified & eIdas \parencite{adobesign2018legal} \\ \hline
		Easy usage & \\ \hline
		Costs & Enterprise plan required, but no price published online \parencite{adobesign2018costs} \\ \hline
		Accepted documents & PDF, Microsoft Word \parencite{adobesign2018info} \\ \hline
		Amount of accounts & Unknown \\ \hline
	\end{tabular}
	\caption{Summary Adobe Sign}
	\label{tab:adobesign}
\end{table}

\subsection{SignNow}

\begin{table}[h]
	\begin{tabular}{|p{4cm}|p{10cm}|} \hline
		Criterion & Fulfillment \\ \hline
		Usage without license & \\ \hline
		Devices independent & \\ \hline
		Integration possible & \\ \hline
		Certified & \\ \hline
		Easy usage & \\ \hline
		Costs & \\ \hline
		Accepted documents & \\ \hline
		Amount of accounts & \\ \hline
	\end{tabular}
	\caption{Summary SignNow}
	\label{tab:signnow}
\end{table}

\subsection{RightSignature}

\begin{table}[h]
	\begin{tabular}{|p{4cm}|p{10cm}|} \hline
		Criterion & Fulfillment \\ \hline
		Usage without license & \\ \hline
		Devices independent & \\ \hline
		Integration possible & \\ \hline
		Certified & \\ \hline
		Easy usage & \\ \hline
		Costs & \\ \hline
		Accepted documents & \\ \hline
		Amount of accounts & \\ \hline
	\end{tabular}
	\caption{Summary RightSignature}
	\label{tab:rightsignature}
\end{table}

\subsection{eSign Live}

\begin{table}[h]
	\begin{tabular}{|p{4cm}|p{10cm}|} \hline
		Criterion & Fulfillment \\ \hline
		Usage without license & \\ \hline
		Devices independent & \\ \hline
		Integration possible & \\ \hline
		Certified & \\ \hline
		Easy usage & \\ \hline
		Costs & \\ \hline
		Accepted documents & \\ \hline
		Amount of accounts & \\ \hline
	\end{tabular}
	\caption{Summary eSign Live}
	\label{tab:esign}
\end{table}

\subsection{PandaDoc}

\begin{table}[h]
	\begin{tabular}{|p{4cm}|p{10cm}|} \hline
		Criterion & Fulfillment \\ \hline
		Usage without license & \\ \hline
		Devices independent & \\ \hline
		Integration possible & \\ \hline
		Certified & \\ \hline
		Easy usage & \\ \hline
		Costs & \\ \hline
		Accepted documents & \\ \hline
		Amount of accounts & \\ \hline
	\end{tabular}
	\caption{Summary PandaDoc}
	\label{tab:pandadoc}
\end{table}

\subsection{eSignAnyWhere}
