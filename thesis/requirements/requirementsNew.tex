The new business process should solve the problems explained in detail in section \ref{sec:issues}. But additional there are more requirements to the new process in the technical and formal area. An overview about the general targets and their requirements is given in table \ref{tab:overviewTargets}. They are explained afterwards in this chapter. First the targets of the new process are explained. Next the requirements are described and finally a sketch of the new process presented. 

\begin{table}[h!]
	\begin{tabular}{|l|l|c|} \hline
		\rowcolor{Gray}Target & Requirement & Importance \\ \hline
		\multirow{2}{*}{Satisfaction signing guideline} & Checking of fulfillment & 1High \\ \cline{2-3}
														& Transparency & High \\ \hline
		\multirow{3}{*}{Reduce manual work} & Automate document creation & Low \\ \cline{2-3}
											& Automate of signing process & High \\ \cline{2-3}
											& Automate document archiving & Medium \\ \hline
		\multirow{3}{*}{Reduce paper usage} & Accepted file types & High \\ \cline{2-3}
											& Fulfillment of legal standards & High \\ \cline{2-3}
											& Sign most documents electronically & Low \\ \hline
		\multirow{2}{*}{Introduction of a corporate-identity} & One document design & Low \\ \cline{2-3}
											& Sending about defined mail addresses & Medium \\ \hline
	\end{tabular}
	\centering
	\caption{Overview targets and requirements}
	\label{tab:overviewTargets}
\end{table}


\section{Targets}
The company \gls{cc} hops that the new process can solve the problems they currently have. Therefore, they defined the following targets that should be reached with the new process:
\begin{enumerate}
	\item Satisfy the signing guideline. \newline
	In the future all documents should be signed regarding the signing guideline. It should not be possible to send an unsigned or incorrect signed document to the customer. Furthermore, the employees should not think about what they need to sign with which amendment. The system, implementing the new business process, should predetermine the needed fields to be filled in with the information of the according content.
	\item Reduce the manual work. \newline
	The interaction with the sites of \gls{cc} should get faster and simpler. Also, the creation of the documents should be done based on templates to reduce the formal mistakes inside of the documents and give the employees standards for them. Moreover, the manual work should be reduced. In the best case the system automatically fill the standard information correct in and add the fields for signing without any help of the human.  
	\item Reduce the usage of paper in the office. \newline
	Due to the fact that all contracts and belonging documents are either stored electronically or need to be archived in the headquarter at Solingen. This leads to situation that the documents when the archived electronically will be disposed properly or send with the postal way to Solingen and placed there in the physical archive. In the case that this will be done mostly everything electronically, the costs will be reduced for the properly disposal, sending and the maintenance and the place of the physical archive. 
	\item Introduction of a corporate-identity: \newline
	At the moment exist no corporate-identity within the different documents due to the fact that they are created inside different tools and by several templates. In the future they should have the same style (color, font, size, etc.) and basic layout.
\end{enumerate}
The result should be that \gls{cc} can work more effectively so that the reaction time by customer requests can be speed up and unnecessary and error-prone work can be reduced due to automation by fulfilling given regulations from the government and the executive board of \gls{cc}.

\section{Requirements}
For the new process a few requirements need to be fulfilled to reach the targets defined previously and the acceptance of all employees and customers of \gls{cc}.
\begin{enumerate}
	 \item Automated control about fulfillment of the signing guideline: \newline
	 The new process should make it simple to fulfill the signing guideline. In the best case all persons need to sign, based on the signing guideline should be automatically invited to sign in the correct order. This functionality requires an automatic insertion of the needed data fields like date, place and signature. With the automation it is ensured that the signing guideline is always fulfilled due to the fact that the creator of the document do not need to know who had to sign from \gls{cc}.
	 \item Transparency of the signing guideline: \newline
	 Through the new process the signing guideline should get more transparent for the staff of \gls{cc}. This means during the creation of a document the employee should get the information if he need to inform a person from the executive board. Moreover, by the creation of the new process clear terminology definitions should be created to avoid communication and interpretation problems between the staff and the different departments of \gls{cc}.
	 \item Automated document creation: \newline 
	 To avoid errors by manual copying of data from one place to another like ID of the quote or the volume, it should be possible that there are templates for all document types (e.g. quote, contract, invoice) which could be filled in simple cases direct from the system or partly from the system and the human. The templates need to be created with people that know the needs for each document.
	 \item Automation of the signing process: \newline
	 Another point is the signing process. In the best case the required fields should be set automatically and the control about the  procedure should be handled by a system automatically, so that the signer and the initiator of the system could care about other things than coordinating the signature, like sending a mail to remind or select how had to sign. This avoids errors and free time for the initiator.
	 \item Automated archiving of documents: \newline
	 All documents should be archived regarding their content and relation to other documents. This should leads to a clear structure, which can be fast understand and simple to use for the employees working with that documents. At the end the amount of work for searching and archiving should be reduced and the processing of other tasks should get faster cause of the sorted archive.
	 \item Accepted file types: \newline
	 At \gls{cc} currently the following document formats are used to create and store documents: \gls{PDF}, Microsoft Word and Open Office Writer. They should also be used in the new process, to avoid problems with the usage and installation of the new software for document creation. Additional it should not be to complicated too create new documents.
	 \item Fulfillment of legal standards: \newline
	 Another requirement is that the new process need to fulfill legal standards. This includes aspects of data protection, regulations regarding electronic signature and security aspects. Additional it need to keep care that the standards from the customer, partner and the company itself need to be satisfied. \newline
	 As \gls{es} should at least the \gls{aes} be used, because it makes possible to ensure authenticity and integrity as can be seen in appendix \ref{res:es}.
	 \item Sign most documents electronically: \newline
	 In the best case all documents possible should be signed electronically, but therefore a lot of legal requirements need to be satisfied. The most priority at the moment have quotes, \glspl{nda} and contracts with the customer. In the case most documents are signed with the new way, a lot of workload for the backoffice employees is reduced and the costs for the storage and the paper is decreased.
	 \item One document design: \newline
	 To achieve inside the different document types a corporate-identity, a general document design need to be created with a layout, used font style and more aspects. Therefore, the help of designers is required. The design should be used for every document that is used for interactions with customers, potential employees and other situations.
	 \item Sending of documents via predefined mail addresses: \newline
	 Furthermore, it will be helpful when the documents could be sent about one mail address specified per responsibility like quotes and order confirmation with \textit{office@codecentitc.de}. This ensures the correct receiving of an answer from the customers. The implementation establishes a simple response functionality for the customer, because the German government determine, that the authenticity and integrity of a received document with mail need to be guaranteed.
\end{enumerate}
 If all those requirements are realized, the targets should be reached. Due to the fact that they all have different priorities as shown in table \ref{tab:overviewTargets}, they will be processed in a different order than they are listed above.

\section{New Process}
The new process should coordinate the signing of documents. Therefore, the user should be able to search for documents based on the name of it, the type of the document, related customer or on the name of the employee placed the document in the document storage / \gls{erp}. Afterwards the user can select a document from the search result and the system implementing the process checks if the user is allowed to sign the document and if the first check is yes if another person with different position need to sign with the user. If needed the system gives the user the choice to specify the person who had to sign along with the user. When all required information are defined like mail addresses and names of persons involved in the signing process, the system sends them all with the document to the tool for the \gls{es}. Finally, the user need to check if everything is correct and the signing sub process can be started. This should be controlled by the other tool. 

\section{New Signing Guideline}
Since the first April 2018 \gls{cc} has a new signing guideline, which should solve the problems with the old one. In table \ref{tab:newSigningGuideline} it is represented for the documents focused at the bachelor thesis. In the case a `-' is presented in the cell it is not allowed for the person with that position to sign the document.

\begin{table}[h!]
	\begin{tabular}{|p{1.5cm}|p{2cm}|p{2cm}|p{2cm}|p{2cm}|p{2cm}|p{2cm}|} \hline
		\multicolumn{2}{|c|}{Document} & \multicolumn{5}{c|}{Company Position}\\ 
		Type & Value & Employee & Salesman & Manager & Procurator & Chairman \\ \hline
		\multirow{3}{1.5cm}{Quote} & $ \leq 50 000 \euro$ & alone & alone & alone & alone & alone \\ \cline{2-7}
							 & $ 50 000 \leq 400 000 \euro$ & - & with Manager / Procurator / Chairman & alone & alone & alone \\ \cline{2-7}
							 & $ < 400 000 \euro$ & - & with Procurator / Chairman & with Procurator / Chairman & alone & alone \\ \hline
		\multirow{3}{1.5cm}{Contract} & $ \leq 50 000 \euro$ & - & alone & alone & alone & alone \\ \cline{2-7}
									& $ 50 000 \leq 400 000 \euro$ & - & with Manager / Procurator / Chairman & alone & alone & alone \\ \cline{2-7}
									& $ < 400 000 \euro$ & - & with Procurator / Chairman
									& with Procurator / Chairman & alone & alone \\ \hline
		\Gls{nda} & / & - & - & alone & alone & alone \\ \hline
	\end{tabular}
	\centering
	\caption{New Signing Guideline Presentation}
	\label{tab:newSigningGuideline}
\end{table}

For a better overview the process is shown in the image: ToDo 
