The new business process should solve the problems explained in section \ref{sec:issues}. But additionally there are more requirements to the new process in the technical and formal area. An overview about the general goals and their requirements are given in table \ref{tab:overviewTargets}. They are explained afterwards in this chapter. First the goals of the new process are explained. Next the requirements are described and finally a description of the new process and the new signing guideline is presented. 

\begin{table}[h!]
	\begin{tabular}{|l|l|c|} \hline
		\rowcolor{Gray}Goal & Requirement & Importance \\ \hline
		\multirow{2}{*}{Satisfy signing guideline} & Checking of fulfillment & High \\ \cline{2-3}
		& Transparency & High \\ \hline
		\multirow{3}{*}{Reduce manual work} & Automate document creation & Low \\ \cline{2-3}
		& Automate signing process & High \\ \cline{2-3}
		& Automate document archiving & Medium \\ \hline
		\multirow{3}{*}{Reduce paper usage} & Accepted file types & High \\ \cline{2-3}
		& Fulfillment of legal standards & High \\ \cline{2-3}
		& Sign most documents electronically & Low \\ \hline
		\multirow{2}{*}{Introduction of a corporate-identity} & One document design & Low \\ \cline{2-3}
		& Sending about defined mail addresses & Medium \\ \hline
	\end{tabular}
	\centering
	\caption{Overview goals and requirements}
	\label{tab:overviewTargets}
\end{table}


\section{Goals}
The company \gls{cc} hopes that the new process can solve the problems they currently face. Therefore, they defined the following goals that should be reached with the new process:
\begin{enumerate}
	\item Satisfy signing guideline. \newline
	In the future all documents should be signed according to the signing guideline. It should not be possible to send an unsigned or incorrectly signed document to the customer. Furthermore, the employees should not think about what they need to sign with which amendment. The system should predetermine the needed fields that have to be filled in.
	\item Reduce manual work. \newline
	The interaction with the sites of \gls{cc} should get faster and simpler. Also, the creation of documents should be done based on templates to reduce the formal mistakes inside of the documents and give the employees standards for them. Moreover, the manual work should be reduced. In the best case the system automatically fills the default information in and add the fields for signing without any help of an employee.  
	\item Reduce usage of paper in the office. \newline
	Due to the fact that all contracts and belonging documents are either stored electronically or need to be archived in the headquarter at Solingen. This leads to a situation that the documents, when they are archived electronically, will be disposed properly or send by letter to Solingen and placed there in the physical archive. In the case that this will be done mostly electronically, the costs will be reduced for the proper disposal, sending, maintenance and place of the physical archive. 
	\item Introduction of a corporate-identity: \newline
	At the moment no corporate-identity exists within the different documents, due to the fact that they are created inside different tools and by several templates. In the future they should have the same style (color, font, size, etc.) and basic layout.
\end{enumerate}
The result should be that \gls{cc} can work more effectively, so that the reaction time by customer requests can be speed up and unnecessary, error-prone work can be reduced, due to automation by fulfilling given regulations from the government and the executive board of \gls{cc}.

\section{Requirements}
For the new process a few requirements need to be fulfilled, to reach the goals defined previously and the acceptance of all employees and customers of \gls{cc}.
\begin{enumerate}
	\item Automated control about fulfillment of the signing guideline: \newline
	The new process should make it simple to fulfill the signing guideline. In the best case, all involved parties should automatically be invited to sign in the correct order, based on the signing guideline. This functionality requires an automatic insertion of the needed data fields for example date, place and signature. With the automation it is ensured that the signing guideline is always fulfilled, due to the fact that the creator of the document does not need to know who had to sign.
	\item Transparency of the signing guideline: \newline
	Through the new process the signing guideline should get more transparent for the staff of \gls{cc}. This means that during the creation of a document the employee should get the information whether he needs to inform a person from the executive board. Moreover, by the creation of the new process clear terminology definitions should be created to avoid communication and interpretation problems between the staff and the different departments of \gls{cc}.
	\item Automated document creation: \newline 
	To avoid errors by manual copying data from one place to another like ID of the quotation or the volume, it should be possible that there are templates for all document types (e.g. quotation, contract, invoice). In simple cases this could be filled in directly or partially by the system and completed by an employee. The templates need to be created with people that know the requirements of each document type.
	\item Automation of the signing process: \newline
	Another point is the signing process. In the best case the required fields should be set automatically and the control about the procedure should be handled by the system automatically as well. The signer and the initiator of the system could care about other things than coordinating the signature, for example sending a mail to remind or select who had to sign the document. This avoids errors and creates free time for the initiator.
	\item Automated archiving of documents: \newline
	All documents should be archived regarding their content and relation to other documents. This should lead to a clear structure, which is easy to understand and simple to use for the employees working with that document. In the end, the amount of work for searching and archiving should be reduced and the processing of other tasks should get faster due to the sorted archive.
	\item Accepted file types: \newline
	Currently the following document formats are used to create and store documents at \gls{cc}: \gls{PDF}, \textit{Microsoft Words} .docx and \textit{Open Office Writers} .odt. They should also be used in the new process, to avoid problems with the usage and installation of the new software for document creation. Additionally it should not be too complicated to allow new documents formats.
	\item Fulfillment of legal standards: \newline
	Another requirement is that the new process needs to fulfill legal standards. This includes aspects of data protection regulations regarding electronic signature and security aspects. Additionally it needs to care that the standards from the customer, partner and the company itself have to be satisfied. \newline
	As the \gls{es} service the \gls{aes} implementation should be used, because it makes it possible to ensure authenticity and integrity as it can be seen in appendix \ref{res:es}. A short summary is given in the section \ref{sec:researchTool}.
	\item Sign most documents electronically: \newline
	In the best case all documents possible should be signed electronically, but therefore a lot of legal requirements need to be considered. At the moment quotations, \glspl{nda} and contracts with the customer are most important. In this case most documents are signed with the new way, a lot of workload for the backoffice employees is reduced and the cost for storage and paper is decreased.
	\item One document design: \newline
	To achieve a corporate-identity throughout the different document types, a general document design needs to be created with a layout, used font style and more aspects. Therefore, the help of designers is required. The design should be used for every document that is used for interactions with customers, potential employees and other situations.
	\item Sending of documents via predefined mail addresses: \newline
	Furthermore, it will be helpful when the documents could be sent through one email address specified per responsibility like quotations and order confirmation with \textit{office@codecentitc.de}. This ensures the correct receiving of an answer from the customers. The implementation establishes a simple response functionality for the customer, because the German government determines, that the authenticity and integrity of a received document via email needs to be guaranteed.
\end{enumerate}
If those requirements are realized, the goals should be reached. Due to the fact that they all have different priorities as shown in table \ref{tab:overviewTargets}, they will be processed in a different order than they are listed above.

\section{New Process}
The new process should coordinate the signing of documents. Therefore, the user should be able to search for documents by name, type of the document, related customer or the name of the employee that placed the document in the document storage or \gls{erp}. Afterwards the user can select a document from the search result, and the system implementing the process checks if the user is allowed to sign the document. If needed, the system gives the user the choice to specify the person, who has to sign along with the user. When all required information is defined, for example email addresses and names of persons involved in the signing process, the system sends them all together with the document to the tool for the \gls{es}. Finally, the user needs to check if everything is correct and the signing sub process can be started. This should be controlled by the other tool. 

\section{New Signing Guideline}
Since the first April 2018 \gls{cc} has a new signing guideline, which should solve the problems with the old one. In table \ref{tab:newSigningGuideline} the documents that are focused in this project are represented. 
In the table a `-'represents, that the person in that position not allowed to sign the document.

\begin{table}[h!]
	\begin{tabular}{|p{1.5cm}|p{2cm}|p{2cm}|p{2cm}|p{2cm}|p{2cm}|p{2cm}|} \hline
		\rowcolor{Gray} \multicolumn{2}{|c|}{Document} & \multicolumn{5}{c|}{Company Position}\\ 
		Type & Value & Employee & Salesman & Manager & Procurator & Chairman \\ \hline
		\multirow{3}{1.5cm}{Quotation} & $ \leq 50.000 $ \euro & alone & alone & alone & alone & alone \\ \cline{2-7}
		& $ > 50.000 \leq 400.000 $ \euro & - & with Manager / Procurator / Chairman & alone & alone & alone \\ \cline{2-7}
		& $ > 400.000 $ \euro & - & with Procurator / Chairman & with Procurator / Chairman & alone & alone \\ \hline
		\multirow{3}{1.5cm}{Contract} & $ \leq 50.000 $ \euro & - & alone & alone & alone & alone \\ \cline{2-7}
		& $ > 50.000 \leq 400.000 $ \euro & - & with Manager / Procurator / Chairman & alone & alone & alone \\ \cline{2-7}
		& $ > 400.000 $ \euro & - & with Procurator / Chairman
		& with Procurator / Chairman & alone & alone \\ \hline
		\Gls{nda} & / & - & - & alone & alone & alone \\ \hline
	\end{tabular}
	\centering
	\caption{New Signing Guideline Presentation}
	\label{tab:newSigningGuideline}
\end{table}

This signing guideline needs to be fulfilled by the new process. Moreover, there needs to be a storage for all persons regarding their position at the company. 
