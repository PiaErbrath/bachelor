\section{Test Protocol SignNow}
\label{tool:sec:signnow}
This tool is tested based on the requirements defined in the research about tools to sign documents electronically. It is done at the twelfth April 2018 with the test version of the Business Premium plan, which has a duration of six days. Additionally, an \gls{app} exist, which is tested on an Android system.

\subsection{Registration}
To register in the test version only the following data is required: name, mail address and password. After submit that data to the tool a mail verification is done and the user can sign in.

\subsection{Uploading Documents}
On the web application of SignNow only a file chooser exists, in which the user can select the document from a directory at the computer. In the case he often need to sign/ let sign a document, he could make a template out of it, which is stored at SignNow. Inside the \gls{app} more options exist to select a document: make a picture from it, import it from another \gls{app} running on the smartphone/tablet or upload it from the SD-card.  

\subsection{Sharing of Documents}
Another important aspect is the sharing of the documents for signing or notification. Within SignNow it is simple: Just add the name and the mail address of those persons who should sign the document or notified about it. This could also be done when the document is completely signed and need to be shared again without signing. Moreover, it is possible to customize the subject text content of the mail per user. In the \gls{app} it is not possible to share the document.

\subsection{Select Order of Signing}
SignNow has the option to make an order of signing. It is called steps. A nice feature here is that multiple signer are allowed in one step. The order of the steps and the signers could be changed with a drag and droop functionality.

\subsection{Preparation for Signing}
To prepare the document for signing a drag and drop functionality exists. The required fields could be selected per signer. In the case one or more signer are not associated to at least one field, the document could not be stored and an information window appears for that error.

There are several file types. There are two categories: The first can be used always, that means inside templates, free signing or for the initiator of a signing process. The second ones are only available when a new document is uploaded and could only assigned to those signer which are invited with a mail. These are they all:
\begin{enumerate}
	\item Signature, date of signing, text and check box
	\item Initials, request for attachments (additional documents/pictures), drop down menu to select one option or radio button to vote one of different options.
\end{enumerate}

\subsection{Company Stamp}
SignNow do not offer an option to add a company stamp. The field type request for attachments could be reused for that, but this is not a perfect solution through the fact, that it will not be integrated inside the document, but in a compromised directory.

\subsection{Signing Documents}
The signing is simple inside SignNow. Only click on the filed for the signature and four options appears in a window:
\begin{enumerate}
	\item Type: The user type his name in the defined field and the tool generates several styles based on the name. Afterwards the user select one of the style as his signature.
	\item Drawing: The user draws the signature with the mouse or the touch pad.
	\item Upload: The user uploads an image of his signature from a storage.
	\item My signature: In the case the user had already signed with SignNow, the used signature from that is stored and can be reused.
\end{enumerate}
In the \gls{app} only the last three options are available. In the complete process the signers do not need to agree on accepting the \gls{es} by default.

The process initiator can directly sign within the step of the document preparation, all other receives a mail or get it shown in their view if they have an account.
\subsubsection{One Person}
In the case only one person need to sign, it depends on how had to sign. If the initiator is the person, he directly can sign with opening the document, add the fields and store it. Afterwards he could share the document via mail. Else just the mail address need to be added, options and fields set and then it need to be sent. The signer click on the link or select it in the view of the tool if he has an account, fill in the data and click on done. In the step of filling the data fields he is guided. The signing is also possible within the \gls{app}.

\subsubsection{Multiple Persons}
In the case of multiple signatures are required, the tool controls that all receives the invitation of signing, that there are no multiple signatures are at the same time and that are all signed based on the order if set. The process of signing is for each person the same as described in the part for one person above. 

\subsubsection{Unregistered Person}
It is also possible to sign a document even the person does not have an account. He can generate a signature as described above. In the case he already signed a document with SignNow, the tool stored his signature. 

\subsubsection{Decline Signing}
A person, who is asked to sign a document, can also decline this request due to select the option by viewing the document. After the selection he can enter a reason and this information will sent to all involved parties. In  the case he later changes his opinion he can sign it then, if the document is not deleted by the owner.

\subsubsection{Features}
SignNow offers some functionalities not requested in the test scenario, which are tested as well. They are described below.

\paragraph{Delegate signing}
In the case the initiator do not enable the option it is possible to forward the document to someone else to sign. Therefore, only the name and the mail address need to be entered. If this option is not disabled, no security level could be added.

\paragraph{Free signing}
Another feature is to send a document for signing to some other, but do net set any field to be filled by others. The other party can decide which fields to set where. The only requirement is a signature field. After the document is signed by all parties, all receives the completed document.

\paragraph{Signing link}
This feature is similar to the free signing. The differences are the following: First it is only for templates (stored documents that are defined as templates and often used) and a link is generated, which can be sent to all persons that should sign the template.

\paragraph{Security aspects}
The tool offers three different security methods. Only one of them could be tested through the fact that the others are only available for user with an American phone number.
\begin{enumerate}
	\item Password: To sign a document a password need to be entered before. Therefore, it will be added per mail address in the preparation of the sending. It need to be submitted to those persons in some way.
	\item \gls{sms}: This option was not testable. The initiator need to enter a phone number and the other person receives an \gls{sms} with a password, which need to be entered before signing.
	\item Phone call: This option was not testable and no description was found.
\end{enumerate}

\subsection{Status Report}
In the case the user has an account he can see the status within the web application or in the \gls{app}, if he installed on the smartphone/tablet. Otherwise it is possible to set the option to receive status information with mails. This is than possible for all person assigned to the document.

\subsection{Availability of Document}
After the signing the document is available for all persons, involved in the signing process, having an account at SignNow on the web application or the \gls{app}. Moreover, all people receive a mail with the complete signed document and in the case files were attached also them. 

\subsection{Personal Impression}
The tool looks different in the browsers. In Firefox, they look better and gives more guidance, than in Google Chrome or Internet Explorer. But generally the tool is most of the time intuitive and a user manual is online available. 