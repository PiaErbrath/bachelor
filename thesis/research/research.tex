Inside this chapter an overview about the two researches, which are made during the project, is given. First of all the research about electronic signature is explained, where general information was collected for the project. The second research is about electronic signing tools, which was the main research for the project. And finally the changed requirements of the project are listed.

\section{Research about Electronic Signature}
Within the first weeks of the bachelor project the research about electronic signature was created. It is added in the appendix \ref{res:es}. This section will give a short overview about the scope, criteria, how it was executed and the result of it.

\subsection{Approach}
The aim of this research was to gain knowledge about \gls{es} in general. \Gls{cc} uses an \gls{es} at the moment, but there was not a clear regulation about its usage. This should also be clarified in the research. Included were aspects like the legal situation in Germany and the \gls{eu}. Moreover, the different types of \gls{es} should be analyzed and an advice should be given which one to use. Therefore, different questions were set up and criteria were defined together with the employees of \gls{cc}. At the end an advice should be given whether \gls{es} should be used at \gls{cc} and when the advice is positive, which \gls{es} type should be used.

\subsection{Strategy}
To fulfill the approach an internet research was done. Governance documents, papers from companies that provide tools for electronic signature and lawyer blog posts have been found. The information are taken into account based on the date they were written (some of them were outdated by newer legal changes) and credibility of the author.


\subsection{Result}
Since the year 2014 an \gls{eu} wide regulation for \gls{es} exists, called \gls{eidas}, which needs to be fulfilled by the 1st July 2016 in all \gls{eu} member states and the \gls{eea} \parencite{BundesministeriumInneren2018,Steffens2018eIDAS}. Also, in most other countries in the world the \gls{es} is accepted by governance. 

Moreover, exist four different \gls{es} types:
\begin{enumerate}
	\item Simple electronic signature: The easiest \gls{es}, but also with the lowest provability. Can be the typing of the name at the end of a document or inserting an image with the signature \parencite{eIDAS2014,CEFd2018}.
	\item Advanced electronic signature: More complex \gls{es} type, which guarantees that no changes are made to the document after it is signed and identified by the signer \parencite{eIDAS2014}.
	\item Qualified electronic signature: In general the same as \gls{aes}, but with provability and certification by governance instances. Mostly requires additional hardware, but there are solutions on the market, which do that inside a cloud service \parencite{eIDAS2014,CEFd2018}.
	\item Biometric signature: This type is at the moment not explained in detail and is not accepted by governances. There are two sub categories with different approaches: static and dynamic. Both are good to identify the signer, but depending on the implementation it is not possible to detect document changes. Furthermore, additional hardware is mostly required for the signing process.
\end{enumerate}

After all it would be advisable for \gls{cc} to use either the \gls{qes} or \gls{aes}, because both identify the signer and detect document changes after signing. Their points have only one value difference, see table \ref{es:Tab:comp} for further details. And regarding usability and costs factors, the \gls{aes} fits for the standard use cases at \gls{cc} and the legal standards.

\section{Research about Electronic Signing Tool} \label{sec:researchTool}
To introduce a new process for signing documents electronically there are two options: Either the \gls{es} tool is created or an already existing tool is used for \gls{es}. In relation to the requirements for \gls{es} tools and their acceptance by the customers of \gls{cc}, it was not an option to develop a new solution. The major reasons are the time constraints and the high expectations that need to be fulfilled. Therefore, the decision was made to use an existing tool. This results in a research for tools that could be used. The complete research is placed in the appendix \ref{res:tool}. 

\subsection{Approach}
Within the research a tool should be selected that best fits the requirements of \gls{cc}. They are influenced by financial and programmatic aspects. To figure them out questions were created and needs of \gls{cc} are documented from the new process. Then a selection of tools was made and tested. Tools will be weighted and in the end a conclusion is made to decide on one tool.  


\subsection{Strategy}
First of all, the criteria were defined together with members of \gls{cc}. Based on that a selection of tools was made that are taken into account. Therefore, partners of \gls{cc} were requested if they have a solution, suggestions from employees are collected and web comparison portals are used to identify potential candidates. Afterwards a scenario was created to test all of them on usability and ease of use. In this case additional information are required they are either collected from the websites of the provider, related blogs or from interviews with the sales departments of them. In those interviews demos were given and an analysis of the required functionalities was made.  

\subsection{Result}
This research has nine potential candidates for the tool: \textit{Wacom} technology for electronic biometric signature, \textit{DocuSign}, \textit{HelloSign}, \textit{SignDoc}, \textit{AdobeSign}, \textit{SignNow}, \textit{eSign Live}, \textit{PandaDoc} and \textit{eSignAnyWhere}. Most of them have a similar functionality, only \textit{Wacom} and \textit{SignDoc} were not really testable, because of hardware problems and are not taken into account in the final comparison. Furthermore, \textit{AdobeSign} is not in the final comparison, because of missing information despite repeated demand.

In the end \textit{DocuSign}, \textit{HelloSign} and \textit{eSignAnyWhere} are the tools with the most points and are closely ranged. \textit{HelloSign} has the most points, followed by \textit{DocuSign} and then \textit{eSignAnyWhere}. But it is advisable to use \textit{DocuSign}, because there is a German-speaking support available, it has a higher security standard and all requested functionalities are available. However, it is a bit more expensive.  

\section{Change of Requirements}
During the time the research about \gls{es} tools was created, it was revealed that the current \gls{erp} will be replaced through a new \gls{erp} till the end of the year 2018. Therefore, a complete set of requirements will be engineered and this has already started during the second phase of the project. Moreover, the fact was given, that the end of the project is the earliest date, where it will be clear which new \gls{erp} is introduced at \gls{cc}. But due to the fact that using the \gls{api} of the signing tool will cost money it was decided to not buy such a license. Moreover, there are no information available about the costs of the \gls{api} of the current \gls{erp} \textit{ScopeVisio}. Instead of creating a complete system, a prototype should be created with the developer \gls{api} of the tool \textit{DocuSign}, but the system should be created in an exchangeable way. The \gls{erp} should be simulated.