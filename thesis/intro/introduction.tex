This document is the report about the bachelor thesis of Pia Erbrath. The author is a student from the Fontys Hogeschool of Techniek en Logistiek at the site in Venlo, Netherlands. The following chapter gives an overview about the context, the company the bachelor thesis is done and the document structure.

\section{Context}
At the end of the study, which have a duration of eight semesters, a bachelor thesis need to be created. This is to be done in a company, to get \flqq real\frqq life experiences and to can apply his knowledge gathered during the study.  

\section{codecentric AG}
\Gls{cc} was founded 2005 in Solingen and is focused on the agile development of software and the usage of innovative technology \parencite{codecentric2018unternehmen}. The products distributed by \gls{cc} are: 
\begin{itemize}
	\item Services in \gls{IT}-Technology
	\item Consultancy service for software and performance
	\item Software development
	\item Training and workshops for developed software
\end{itemize}
At the moment \gls{cc} have fifteen sites in Europe and about 400 employees. Also, \gls{cc} has two subsidiary start-ups \glspl{cd} and Instana \parencite{codecentric2018startups} and a lot of partners e.g. Scrum.org or elastic \parencite{codecentric2018partner}.
That leads to many competences with new technology like \textit{Internet of Things}, \textit{Big Data}, \textit{Continuous Delivery} and more.
A lot of customers from several areas and size use these competences and products.

In Solingen is the head quarter of \gls{cc} with space for 200 employees. Different departments are located her, like Finance, Sales, \gls{ds} and a team \gls{cd}.   
The bachelor thesis is executed in the department \gls{ds}, which has currently a size of around fifteen members.

The project is for the administration from \gls{cc}. That includes mostly the departments Finance, Sales and Human Resources, but also other departments could be influenced through the project.

\section{Document Structure}
Inside this section the document structure will be explained. Behind the introduction first the project assignment is described followed by the project management. Next an analysis of the current state is made. Then the requirements of the new process is defined. And finally a conclusion with a lookup is given.