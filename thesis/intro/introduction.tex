This document is the report about the bachelor thesis of Pia Erbrath. The author is a student from the Fontys Hogeschool Techniek en Logistiek at the site in Venlo, Netherlands. The following chapter provides an overview about the context, the company where the bachelor thesis is elaborated and the document structure. \newline
It is expected that the reader has a basic understanding of software design and development to properly comprehend the chapters design and implementation later in this document. 

\section{Context}
At the end of the study, which has a duration of eight semesters, a bachelor thesis needs to be created. The bachelor thesis is to be done in a company, to get practical work experiences and to apply the knowledge that was\label{is} gathered during the study.  

\section{codecentric AG}
\Gls{cc} was founded 2005 in Solingen and is focused on the agile development of software and the usage of innovative technology \parencite{codecentric2018unternehmen}. The products distributed by \gls{cc} are: 
\begin{itemize}
	\item Services in \gls{IT}
	\item Consultancy service for software and performance
	\item Software development
	\item Training and workshops for developed software
\end{itemize}
At the moment \gls{cc} has fifteen locations in Europe and about 500 employees. \Gls{cc} has also two subsidiary start-ups, \gls{cd} and Instana \parencite{codecentric2018startups}, and a lot of partners e.g. Scrum.org or elastic \parencite{codecentric2018partner}.
That leads to many competences with new technologies like \textit{Internet of Things}, \textit{Big Data}, \textit{Continuous Delivery} and more.
A lot of customers from several areas and sizes use these competences and products.

The headquarter of \gls{cc} is located in Solingen with space for 200 employees. Several departments are located here like Finance, Sales, Document Solution and the team of \gls{cd}.   
The bachelor thesis is executed in the department Document Solution, which has currently a size of fifteen members.

The project is for the administration from \gls{cc}. That includes mostly the departments Finance, Sales and Human Resources, but also other departments could be influenced through the project.

\section{Document Structure}
Inside this section the document structure will be explained. After the introduction, the project assignment is described in chapter \ref{ch:assignment} followed by the project management in chapter \ref{ch:management} and an analysis of the initial situation is given in chapter \ref{ch:analysis}. Then the requirements for the new process are defined in chapter \ref{ch:requirements}. In chapter \ref{ch:research} a short overview about the researches of the project is given, followed by chapter \ref{ch:design} about the design of the created application. Afterwards the implementation is described in chapter \ref{ch:implementation} and thereafter recommendations are given for further implementations in chapter \ref{ch:advice}. Finally, a conclusion is given within chapter \ref{ch:conclusion}.