This document is the report about the bachelor thesis of Pia Erbrath. The author is a student from the Fontys Hogeschool of Techniek en Logistiek at the site in Venlo, Netherlands. The following chapter gives an overview about the context, the company where the bachelor thesis is elaborated and the document structure.

\section{Context}
At the end of the study, which has a duration of eight semesters, a bachelor thesis need to be created. The bachelor thesis is to be done in a company, to get \flqq real\frqq life experiences and to apply the knowledge that is gathered during the study.  

\section{codecentric AG}
\Gls{cc} was founded 2005 in Solingen and is focused on the agile development of software and the usage of innovative technology \parencite{codecentric2018unternehmen}. The products distributed by \gls{cc} are: 
\begin{itemize}
	\item Services in \gls{IT}-Technology
	\item Consultancy service for software and performance
	\item Software development
	\item Training and workshops for developed software
\end{itemize}
At the moment \gls{cc} have fifteen locations in Europe and about 400 employees. \Gls{cc} has also two subsidiary start-ups, \gls{cd} and Instana \parencite{codecentric2018startups}, and a lot of partners e.g. Scrum.org or elastic \parencite{codecentric2018partner}.
That leads to many competences with new technology like \textit{Internet of Things}, \textit{Big Data}, \textit{Continuous Delivery} and more.
A lot of customers from several areas and sizes use these competences and products.

The head quarter of \gls{cc} is located in Solingen with space for 200 employees. Different departments are located here, like Finance, Sales, Document Solution and the team of \gls{cd}.   
The bachelor thesis is executed in the department Document Solution, which has currently a size of fifteen members.

The project is for the administration from \gls{cc}. That includes mostly the departments Finance, Sales and Human Resources, but also other departments could be influenced through the project.

\section{Document Structure}
Inside this section the document structure will be explained. After the introduction the author describes the project assignment followed by the project management and an analysis of the initial situation. Then the requirements of the new process is defined. And finally a conclusion with a lookout of ------------placeholder------------.