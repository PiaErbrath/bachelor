 This chapter gives an overview about the bachelor thesis project. Therefore, the problem of the initial situation and the assignment of the project are explained, the scope and the phases are defined and finally the motivation behind the project is documented.
 
 \section{Problem Description}
 The first plan was to create a \gls{cms} for the customer Wacom, which produces among others the technology and hardware for mobile devices and Windows operating systems that can be used to make graphics or to sign documents \parencite{wacom2018about}. But currently they develop a new \gls{SDK} of their \gls{app} \textit{sign pro PDF}, which allows the signing of documents on a mobile phone with their pen hardware technology\parencite{wacom2018sign}. That will be finished earliest in July 2018 and the \gls{cms} should be build on that \gls{SDK}. Due to the fact that the bachelor thesis project ends at the beginning of July 2018 it is not realizable.
 
 Then a contact to the Finance and Sales department of \gls{cc} was made. They currently want to set up a new \gls{erp}, because the old \gls{erp} does not fit their requirements anymore. One big aspect for them is to reduce the paper work and automate or simplify several processes regarding making quotations, orders and invoices, starting projects and storing data at only one point for all departments. At this point a \gls{cms} would make sense that can interact with the new \gls{erp}.

 \section{Assignment Description}
 The idea of the task is to introduce the \gls{es} with a general usage by \gls{cc}. Currently they use the \gls{es} sporadic, but there are a lot of situation left where the \gls{es} could also be used.
 This means that all possible contracts, quotations and \glspl{nda} should be signed with the \gls{es} and stored digital, so that the paperless office can be created. Included in the task is the creation of a workflow, that automates the process of subscribing regrading the signing guideline of \gls{cc} so that costs for managing could be reduced and the response time is increased. Due to the fact that \gls{cc} introduce a new \gls{erp} this year, it will be only a proof-of-concept, with a mocked \gls{erp}.
 
 In the case that there is time left the creation of contracts and quotations based on templates and existing customer information should be added to the workflow. Moreover, the archiving of contracts regarding the regulations based from the governance and \gls{cc} should be automated. In the case already existing tools should be used for the realization of the project.
 
 \section{Scope}
 The focus is the creation of a process for the electronic signing of documents. Therefore, a system needs to be created that connects an \gls{erp} with a tool that coordinates the document signing. Additional the writing of the bachelor thesis is part of this focus. Only in the case that time is left until the end of the project additional functionality should be implemented like archiving the signed documents and the template creation of contracts and quotations. The thesis concentrates on these aspects of a \gls{cms}, because the development of a complete \gls{cms} will take more time than given through the internship. 
 
 Out of scope are all other document types than quotations, contracts and \glspl{nda}. They will be focused after the internship, also due to the fact that therefore more legal questions need to be clarified. 
 
 \section{Phases \& Products} \label{sec:phases}
 There are five phases in the project:
 \begin{enumerate}
 	\item Project planning
 	\item Analysis of the initial situation and the requirements
 	\item Research about used tools 
 	\item Implementation and testing of the system based on the defined scope
 	\item Finalize bachelor thesis
 \end{enumerate}
 
 Inside these phases several products should be delivered. They are listed below:
 \begin{itemize}
 	\item Project plan
 	\item Documentation of the initial situation
 	\item Research about \gls{es} and tools that could be used to sign documents electronically
 	\item A system to proof the concept, with a documentation
 	\item A report about the bachelor thesis (this document)
 \end{itemize}
 
 \section{Motivation}
 There are different motivation aspects behind the project. First of all is the reducing of work. When the steps of creating, signing and archiving of documents are automated the employees could work more effective. Next the response time gets faster regrading the concluding of contracts. And finally the guidelines of \gls{cc} must be complied by the employees due to the fact that there is an automated process based on the guidelines. 