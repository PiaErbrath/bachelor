At the beginning of the project the plan was to create a \gls{cms} with the possibility to sign documents electronically. 
In the first planning a different customer, \textit{Wacom}, was bet on. That changed, because they did not finish the development of the \gls{sdk} the project should work with. Then it was decided to work for \gls{cc}, because at this moment they do not have a \gls{cms} and they wanted to increase the usage of \gls{es}.

Within their current process they have several problems. The biggest aspects are the amount of manual work that needs to be done with the documents and the issues with the fulfillment of the outdated signing guideline. \newline
To remove the problems, a new process should be created that fulfills the goals of the new signing guideline and reduces manual work and paper usage at \gls{cc}. To introduce and assist the employees to get familiar with the new process, an application should be created, that helps to fulfill the signing guideline with an \gls{es} and coordinates the document storage of signed documents.

Due to the circumstances of high license costs for the signing tool usage and the introduction of a new \gls{erp}, the decision was made to create a prototype, to simulate and check if the application will support the new process. Moreover, it is possible to build the prototype in such a way that it can be combined with any \gls{erp} a company has, as long as it has an \gls{api} to communicate with. It should be as flexible as possible.

During the implementation phase time constrains occurred, that lead to the fact that most functionalities required for simulating the new process are not implemented. It is advisable to pursue development and regularly check the value of the system. But therefore \gls{cc} should wait until they know which new \gls{erp} they get, because there exist a lot of signing tools that are available as plug-in for some \glspl{erp}. It will reduce costs regarding the development and deployment of an own application when they can be connected easily. The only aspect, which needs to be taken into account, is that they may need to make a separate plug-in for checking the signing regulations based on the signing guideline. \newline
In the case no plug-in exists for the new \gls{erp}, the created prototype should be completed and tested for the production at \gls{cc}.

In general it needs to be said, that this bachelor project was not executable as planned due to external circumstances. 
But this could happen with every project that is executed in the working environment of a company.