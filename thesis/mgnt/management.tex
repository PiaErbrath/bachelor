For the managing of the bachelor thesis project different aspects need to be taken into account, like how to do the project in which time and how to ensure a good quality of products. These things are explained in this chapter.

\section{Approach}
The project in general is done in a waterfall model, based on the phases explained in section \ref{sec:phases}. Its wokflow is the following: Only if the current phase is finished, the next phase can be started. In the phases itself another working model could be used. For this project it is only done in the implementation and testing phase. There a customized version of the agile approach \textit{Scrumban} is used.

-------ToDo---------------

The usage of an agile way helps to develop a product that fulfill all requirements specified and is maintainable. This happen as a result of the continuous implementation of additional functionalities in existing basic functionalities. 

For the agile way iterations, also called sprints, are planned with one use case, which is selected together with the leader of the department \gls{ds} and one member of the department sales. 

\section{Time Planning}
The project implementation will be done in an agile way, that means for the implementation of the worflow user stories will be created, and they need to be fulfilled. The preparation that needs to be done before are made inside a waterfall model. A general time planning can be seen in table \ref{tab:timeplanning}. 
\begin{table} [h]
	\centering
	\begin{tabular}{|c|c|l|c|} \hline
		\rowcolor{Gray}Phase & Week & Activity & CW \\ \hline
		\multirow{2}{*}{1} & 1-4 & Project planning, Topic introduction & 5-9 \\
		& 4 & Research \gls{es} & 9 \\	\hline	
		Milestone I & 06/03/2018 & Project Plan & 10\\ \hline
		2 & 5,6 & Analysis current state \& requirements & 10-11  \\ \hline
		3 & 7 & Research tools & 12 \\ \hline
		Milestone II & 27/03/2018 & Midterm report & 13\\ \hline
		3 & 9 & Research tools & 13 \\ \hline
		Milestone III & 05/04/2018 & Midterm presentation & 14 \\ \hline
		4 & 10,11 & Designing new workflow & 14-15 \\
		\multirow{2}{*}{4} & 12-17 & Implementing system & 17-21 \\
		& 18 & Testing system & 22 \\ \hline
		5 & 19 & Final preparation of report \& presentation & 23 \\ \hline
		Milestone IV & 12/06/2018 & Final Report & 24\\ \hline
		Milestone V & 04/07/2018 & Final presentation & ? \\ \hline
	\end{tabular}
	\caption{Time planning of project}
	\label{tab:timeplanning}
\end{table}

\section{Quality Assurance}
The project consist out of three major deliverable categories that have different quality approaches. They are listed below:
\begin{enumerate}
	\item Documents: \newline
	To create qualitatively high documents two techniques are used. On the one hand the writing is done with a tool that checks the orthographic and the grammar and on the other one is the document reviewed by other persons to check the logic and understandability. 
	\item Diagrams: \newline
	In the analysis and the design phase diagrams need to be created. First of all the used diagram types have defined standards, these standards has to be fulfilled. Therefore, tools will be used that can handle the checking. Additional the logic will be checked by persons that currently work (or will work) in the processes visualized in the diagram. 
	\item Implementation: \newline
	For the implementation different use cases will be created, that need to be implemented. The checking will be that they can be executed without errors. Furthermore, if code is created, it will be done based on the defined standard of the used programming language and test driven. Moreover a definition of done will be created, that ensures the quality of the developed code.
\end{enumerate}
The created results will be regular presented and discussed with involved persons. That leads to the possibility to avoid big problems, cause of their early detection. The quality management is agreed with the company. 

\section{Deliverables}
At the end of the internship the following things need to be delivered:
\begin{itemize}
	\item Project Plan: Inside this document the first planning of the bachelor project is written down. The reason of the existence of this document, is to get an overview of the bachelor thesis project for all involved parties. The parties are listed at the information page.
	\item Visualization of the current process: To get an overview about the current process of the sales department a visualization of their process is done with \gls{bpmn} diagrams. Based on them an analysis is done (see chapter \ref{ch:analysis}). The diagrams are placed in the appendix \ref{bpmnOld}.
	\item Research documents: During the bachelor thesis two researches need to be done. The first one is about \gls{es}, which is placed in the appendix \ref{res:es}. It was done to get knowledge about \gls{es} and to clarify which \gls{es} could be used at \gls{cc}. The other research is about tools, which implement the \gls{es} of documents already and can be used for the system to be created during the bachelor thesis. The research document is in the appendix \ref{res:tool}.
	\item Design visualization of the system: To get an understanding of the created system and show the skills of the student, the design of the system should be visualized by several diagrams with \gls{uml} standards, like a class diagram or activity diagrams. The diagrams should present the structure of the implementation and the processes of the systems.
	\item Documentation of the system: Another deliverable is the documentation of the system. This includes user manual, administrator guide and developer documentation (source code, architecture description, etc.). They should give help to work with the later system in the case the author is not available anymore after finishing the project.
	\item Report of thesis: To give an overview about all things done and decisions taken during the bachelor thesis project a document is to be created, which will be read through the examiners of the bachelor thesis. Moreover, the document will be later public available.   
\end{itemize}

\section{Risk Management}
Inside this section risks will be presented in the table \ref{fig:risks}.
\begin{center}
	\begin{landscape}
		\begin{table}[h]
			\begin{tabular}{|p{0,5cm}|p{2cm}|p{4cm}|p{4cm}|p{4cm}|p{1,5cm}|p{1,5cm}|p{1,5cm}|} \hline
				\rowcolor{Gray} \# & Risk & Description & Trigger & Precaution & Probability & Impact & Status \\ \hline
				1 & Time Issue & Long absence from project leads to time problems to finish tasks. & Illness, accident, holidays, to less project boundaries, ... & Estimate more time than expected (time buffer), clear scope definition & 7 & 6 & Occurred \\ \hline
				2 & Legals & Legals influence the requirement of a project and need to be fulfilled otherwise it could results in punishments & New laws from governance/ \gls{eu} & Collaboration a lawyer / person with knowledge about laws, Make research about laws regarding used technology & 2 & 7 & Open \\ \hline
				3 & New technology / programming language / \gls{api} / \gls{SDK} / framework & The usage of new technology results	in unknown problems that could not be solved with that technology due to the fact that requirements could not be fulfilled & Not enough knowledge about 	the used technology & Good research about technology, usage of tools knowledge already exists in company, estimate more time & 3 & 3 & Occurred \\ \hline
				4 & Communi- cation & Cause of the influence of many parties on the project, communication problems could occur, like meetings invitation, incorrect requirements, ... & Missing or incomplete communication & Research about all involved parties, keep them informed and make regular meetings & 3 & 3 & Open \\ \hline
				5 & \gls{erp} & Currently the Backoffice/Finance/\gls{hr} want to switch to a new \gls{erp},
				but at the moment it is not clear which	one will be used. Some steps of	the new workflow depend on the \gls{erp}, so that the workflow could not be implemented at the end	of the project. & Late decision, no possible to fit workflow to the \gls{erp} & Communication, get information about the current state & 5 & 5 & Occurred \\ \hline
			\end{tabular}
			\caption{Risk Register}
			\label{fig:risks}
		\end{table}
	\end{landscape}
\end{center}
